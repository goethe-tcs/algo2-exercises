% !TeX spellcheck = de_DE
\documentclass{uebung_cs}
\usepackage{algo221}
\blattname{Übungen zum Start ins neue Semester}

%%%%%%%%%%%%%%%%%%%%%%%%%%%%%%%%%%%%%%%%%%%%%%%%%%%%%%%%%%%%%%%%%%%%%%%%%%%%
\begin{document}
Mit diesen Aufgaben kannst du deine Fähigkeiten aus \emph{Algorithmen und Datenstrukturen 1} auffrischen. Wenn dir manche dieser Aufgaben schwerfallen, ist es wichtig, die Konzepte nachzuarbeiten!

\begin{aufgabe}[Orga]
    Lies alle Informationen auf der Webseite des Kurses und auf Moodle.
\end{aufgabe}

\begin{aufgabe}[Vereinigung und Schnitt]
    Gegeben sind zwei sortierte Listen $A$ und $B$ von ganzen Zahlen.
    \begin{enumerate}
        \item Beschreibe einen Algorithmus, der die Menge $C=A\cup B$ berechnet. Hierbei soll $C$ wieder eine sortierte Liste sein.
        \item Beschreibe einen Algorithmus, der die Menge $C=A\cap B$ berechnet. Hierbei soll $C$ wieder eine sortierte Liste sein.
    \end{enumerate}
\end{aufgabe}

\begin{aufgabe}[Coderunner]
    Löse die Aufgabe \enquote{Coderunner ausprobieren} auf Moodle.
\end{aufgabe}

\begin{aufgabe}[Pfannkuchensortierung]
    Gegeben ist ein Stapel von $n$ Pfannkuchen verschiedener Größe.
    Du musst die Pfannkuchen der Größe nach sortieren, sodass am Ende kleinere Pfannkuchen auf größeren Pfannkuchen liegen.
    Die einzige erlaubte Operation ist ein \emph{flip}: Führe einen Pfannenwender unter die obersten $k$ Pfannkuchen für eine natürliche Zahl~$k$ ein und wende sie um.
    \begin{enumerate}
        \item Beschreibe einen Algorithmus, der einen beliebigen Stapel von $n$ Pfannkuchen mit $O(n)$ flips sortiert. Wie viele flips braucht dein Algorithmus im worst-case genau?
        \item Für jede positive Zahl $n$, beschreibe einen Stapel mit $n$ Pfannkuchen, auf dem dein Algorithmus $\Omega(n)$ flips benötigt.
    \end{enumerate}
\end{aufgabe}

\begin{aufgabe}[Netzwerke]
    [to-do]
    \begin{enumerate}
        \item a
        \item b
    \end{enumerate}
\end{aufgabe}

Puzzle of the week: 99 Cops

\end{document}
