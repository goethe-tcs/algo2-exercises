% !TeX spellcheck = de_DE
\documentclass{uebung_cs}
\usepackage{algo221}
\uebung{14}{}{}
\blattname{Übungen zu Woche 14: Algorithmen für NP-schwere Probleme}

\usepackage[ruled]{algorithm2e}
\usepackage{cancel}
\usepackage[smaller]{acronym}
\newacro{LP}{Linear Program}
\newacro{ILP}{Integer Linear Program}

%%%%%%%%%%%%%%%%%%%%%%%%%%%%%%%%%%%%%%%%%%%%%%%%%%%%%%%%%%%%%%%%%%%%%%%%%%%%
\begin{document}

Das Übungsblatt enthält alle empfohlenen Lernaktivitäten für die aktuelle Woche.

\begin{itemize}
\item \textbf{Heimarbeit bis Montag 17:00.}
    \begin{itemize}
    \item 
    Schau die Videos an und lies die Buchkapitel.
    \item Bearbeite die \emoji{seedling}-Aufgabe in \href{https://moodle.studiumdigitale.uni-frankfurt.de/moodle/course/view.php?id=2241}{Moodle}. (Feste Abgabefrist!)
    \item Lese den Aufgabentext aller Übungsaufgaben.
    \end{itemize}
\item \textbf{Heimarbeit.} Bearbeite die Übungsaufgaben soweit möglich. Probier zumindest alle mal!
\item \textbf{Dienstag/Donnerstag.}
\begin{itemize}
    \item \textbf{8:00--8:15.} Besprechung im Zoom.
    \item \textbf{8:15--9:15.} Bearbeite jetzt die Übungen, die du noch nicht lösen konntest. Sprich mit anderen Studis! Frag das Vorlesungsteam um Hilfe!
    \item \textbf{9:15--9:45.} Lösungsspaziergang zu den Aufgaben für heute.
\end{itemize}

\item \textbf{Heimarbeit bis Freitag, den 11.02., 17:00.} Gib deine Lösungen zu der \emoji{star}-Aufgabe von diesem Übungsblatt in \href{https://moodle.studiumdigitale.uni-frankfurt.de/moodle/course/view.php?id=2241}{Moodle} ab. (Feste Abgabefrist!)
\end{itemize}

\section*{Dienstag}

\begin{aufgabe}[Gieriges Vertex-Cover]\
	% Erickson Extended Dance Remix - Chapter J, Exercise 2a,b
	\begin{enumerate}
		\item Finde den kleinsten Graphen (minimale Anzahl an Kanten), für den \textsc{GreedyVertexCover} \emph{nicht} die kleinste Knotenüberdeckung ausgibt.
		\item Beschreibe für alle ganzzahligen Werte von $n$ einen Graphen mit $n$ Knoten, für den \textsc{GreedyVertexCover} eine Knotenüberdeckung der Größe $\mathrm{OPT} \cdot \Omega(\log n)$ ausgibt.
		(Hierbei ist $\mathrm{OPT}$ die Größe der kleinsten Knotenüberdeckung.)
	\end{enumerate}
	
\end{aufgabe}

\begin{aufgabe}[Dummes Vertex-Cover]\
	% Erickson Extended Dance Remix - Chapter J, Exercise 3a,b
	\begin{enumerate}
		\item Finde den kleinsten Graphen (minimale Anzahl an Kanten), für den \textsc{DumbVertexCover} \emph{nicht} die kleinste Knotenüberdeckung ausgibt.
		\item Beschreibe eine unendlich große Familie von Graphen, für welche \textsc{DumbVertexCover} eine Knotenüberdeckung der Größe $2 \cdot \mathrm{OPT}$ ausgibt.
		(Hierbei ist $\mathrm{OPT}$ die Größe der kleinsten Knotenüberdeckung.)
	\end{enumerate}
\end{aufgabe}

\section*{Donnerstag}


\begin{aufgabe}[Tiefengesuchtes Vertex-Cover]
	% Erickson Extended Dance Remix - Chapter J, Exercise 5a,b,c
	Betrachte folgenden heuristischen Algorithmus, der in einem zusammenhängenden Graphen $G$ eine Knotenüberdeckung berechnet: 
	\begin{enumerate}[1.,leftmargin=5em]
		\item
		Berechne einen Tiefensuchbaum~$T$ in~$G$, dessen Wurzel ein beliebiger Knoten ist.
		\item
		Gebe die Menge aller Knoten aus, die keine Blätter in~$T$ sind. (Also die Wurzel und alle inneren Knoten von $T$.)
	\end{enumerate}
	\begin{enumerate}
		\item Beweise, dass diese Heuristik eine Knotenüberdeckung in~$G$ findet.
		\item Beweise, dass diese Heuristik eine $2$-Approximation zur minimalen Knotenüberdeckung von $G$ ist.
		\item Beschreibe eine unendlich große Familie von Graphen, für welche diese Heuristik eine Knotenüberdeckung der Größe $2 \cdot \mathrm{OPT}$ ausgibt.
		(Hierbei ist $\mathrm{OPT}$ die Größe der kleinsten Knotenüberdeckung.)
	\end{enumerate}
\end{aufgabe}

\begin{aufgabe}[Beschränkte Suchbäume für Hitting Set]\
	% KT - Exercise 10.1
	Sei $U = \{1,\dots,n\}$ ein Universum mit $n$ Elementen.
	Seien $B_1,B_2,\dots,B_m$ verschiedene Teilmengen von~$U$.
	Eine Menge $H \subseteq A$ heißt \emph{hitting set} für die Kollektion $B_1,B_2,\dots,B_m$, wenn $H$ mindestens ein Element von jeder Menge $B_i$ enthält. Mit anderen Worten: Für alle $i \in \{1,\dots,m\}$ gilt $H \cap B_i \neq \emptyset$.

	Betrachte das \textsc{HittingSet}-Problem:
	\begin{quote}
		\textsc{HittingSet}\\
		\emph{Eingabe:} Mengen $B_1,\dots,B_m\subseteq U$ und Zahl $k\in \mathbb N$.\\
		\emph{Frage:} Hat die Kollektion $B_1,\dots,B_m$ ein \textit{hitting set} der Größe höchstens~$k$?
	\end{quote}

	Sei $b$ die maximale Größe der Eingabemengen, also $b=\max_i|B_i|$.
	Überlege dir einen \textit{bounded search tree} Algorithmus, der das \textsc{HittingSet}-Problem in einer Laufzeit von $f(k,b) \cdot \poly(n,m)$ löst, wobei $\poly(n,m)$ ein Polynom in $n$ und $m$ ist und $f(k,b)$ eine beliebige Funktion ist, die nur von $b$ und $k$ abhängt.
	Welche Funktion~$f$ und welches Polynom~$\poly$ hat die Laufzeit deines Algorithmus?

	\textit{Hinweis: Das Knotenüberdeckungsprolem \textsc{\upshape VertexCover} ist der Spezialfall $|B_1|=\dots=|B_m|=2$.}
\end{aufgabe}

\section*{Sternaufgabe}

\begin{aufgabe}[\emoji{star}: Beschränkte Suchbäume für Clique]
%	Eigenkreation
	Wir haben in Woche $7$ bereits das Entscheidungsproblem \textsc{Clique} kennengelernt:
	\begin{quote}
		\textsc{Clique}\\
		\emph{Eingabe:} Ein Graph $G$ mit $n$ Knoten und eine Zahl $k\in\mathbb N$.\\
		\emph{Frage:} Enthält $G$ eine Clique mit $k$ Knoten?
	\end{quote}
	\begin{enumerate}
		\item Sei $\Delta$ der Maximalgrad von~$G$.
		Überlege dir einen \textbf{rekursiven} \textit{bounded search tree} Algorithmus, der das \textsc{Clique}-Problem in einer Laufzeit von $f(k,\Delta) \cdot \poly(n)$ löst, wobei $\poly(n)$ ein Polynom in $n$ ist und $f(k,\Delta)$ eine beliebige Funktion ist, die nur von $k$ und $\Delta$ abhängt.
		Beschreibe deinen Algorithmus \textbf{in Pseudocode}.
		\item Beweise, dass dein Algorithmus korrekt ist.
		\item Gib die Funktion $f$ konkret an und beweise, dass dein Algorithmus die gegebene Laufzeitschranke einhält.
		\item Wir wissen aus Woche $7$, dass \textsc{Clique} \NP-schwer ist. Impliziert dein Algorithmus aus Aufgabenteil a) nun $\P = \NP$? Erkläre deine Antwort in 2--3 Sätzen.
	\end{enumerate}
\end{aufgabe}


\end{document}
