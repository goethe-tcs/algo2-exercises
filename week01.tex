% !TeX spellcheck = de_DE
\documentclass{uebung_cs}
\usepackage{algo221}
\blattname{Übungen zu Woche 1: All Pairs Shortest Paths}

%%%%%%%%%%%%%%%%%%%%%%%%%%%%%%%%%%%%%%%%%%%%%%%%%%%%%%%%%%%%%%%%%%%%%%%%%%%%
\begin{document}

\section*{Dienstag}

\begin{aufgabe}[Orga]
    Lies alle Informationen auf der Webseite des Kurses, auf Moodle, und auf den Organisationsfolien.
\end{aufgabe}

\begin{aufgabe}[Coderunner]
    Lös die Aufgabe \enquote{Coderunner ausprobieren} auf Moodle.
\end{aufgabe}

% \begin{aufgabe}[Pfannkuchensortierung]
%     Gegeben ist ein Stapel von $n$ Pfannkuchen verschiedener Größe.
%     Du musst die Pfannkuchen der Größe nach sortieren, sodass am Ende kleinere Pfannkuchen auf größeren Pfannkuchen liegen.
%     Die einzige erlaubte Operation ist ein \emph{flip}: Führe einen Pfannenwender unter die obersten $k$ Pfannkuchen für eine natürliche Zahl~$k$ ein und wende sie um.
%     \begin{enumerate}
%         \item Beschreibe einen Algorithmus, der einen beliebigen Stapel von $n$ Pfannkuchen mit $O(n)$ flips sortiert. Wie viele flips braucht dein Algorithmus im worst-case genau?
%         \item Für jede positive Zahl $n$, beschreibe einen Stapel mit $n$ Pfannkuchen, auf dem dein Algorithmus $\Omega(n)$ flips benötigt.
%     \end{enumerate}
% \end{aufgabe}

\begin{aufgabe}[Telefonnetzwerk]
    Gegeben sei ein Diagramm mit Schaltzentren und Verbindungen.
    Verbindungen können durch Kabel, Funk, oder Satellit realisiert werden.
    Wir modellieren das, indem jede Verbindungen eine assoziierte Bandbreite und Kosten hat.
    \begin{enumerate}
        \item Gib einen Algorithmus an, der für zwei Schaltzentren $a$ und $b$ den Verbindungsweg ausrechnet, der die kleinsten Kosten verursacht.
        \item Die Bandbreite eines Verbindungsweg ist die kleinste Bandbreite über alle einzelnen Verbindungen, die auf dem Weg auftauchen.
        Gib einen Algorithmus an, der für zwei Schaltzentren $a$ und $b$ den Verbindungsweg ausrechnet, der die größtmögliche Bandbreite hat.
    \end{enumerate}
\end{aufgabe}

\begin{aufgabe}[Puzzle der Woche: 99 Polizisten]
    Eine Stadt hat 99 Polizisten.
\end{aufgabe}

\section*{Donnerstag}

\end{document}
