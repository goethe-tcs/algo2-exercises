% LTeX: language=de_DE
\documentclass{uebung_cs}
\usepackage{algo223}
\uebung{10}{}{}
\blattname{Übungen zu Woche 10: Berechenbarkeit}

\usepackage[ruled]{algorithm2e}
\usepackage{cancel}

%%%%%%%%%%%%%%%%%%%%%%%%%%%%%%%%%%%%%%%%%%%%%%%%%%%%%%%%%%%%%%%%%%%%%%%%%%%%
\begin{document}
\emph{Zur Erinnerung:}
\begin{itemize}
	\item $\textsc{Accept}(M) := \{w \in \Sigma^{\ast} \;|\; M \text{ akzeptiert } w\}$
	\item $\textsc{Reject}(M) := \{w \in \Sigma^{\ast} \;|\; M \text{ verwirft } w\}$
	\item $\textsc{Halt}(M) := \textsc{Accept}(M) \cup \textsc{Reject}(M)$
	\item $\textsc{Diverge}(M) := \Sigma^{\ast} \backslash \textsc{Halt}(M)$
\end{itemize}

\section*{Dienstag}
\begin{exercise}[Turingmaschinen][\atschool\easy]\
	% Erickson Models of Computation - Chapter 7, Exercise 1
	Sei $M$ eine beliebige Turingmaschine.
	\begin{enumerate}
		\item Beschreibe eine Turingmaschine $M^R$, sodass:
		$$\textsc{Accept}(M^R) = \textsc{Reject(M)} \qquad \text{und} \qquad \textsc{Reject}(M^R) = \textsc{Accept}(M)$$
		
		\item Beschreibe eine Turingmaschine $M^A$, sodass:
		$$\textsc{Accept}(M^A) = \textsc{Accept}(M) \qquad \text{und} \qquad \textsc{Reject}(M^A) = \varnothing$$
		
		\item Beschreibe eine Turingmaschine $M^H$, sodass:
		$$\textsc{Accept}(M^H) = \textsc{Halt}(M) \qquad \text{und} \qquad \textsc{Reject}(M^H) = \varnothing$$
	\end{enumerate}
\end{exercise}

\begin{exercise}[Entscheidbarkeit I][\atschool\medium]\
	% Erickson Models of Computation - Chapter 7, Exercise 3b
	Beweise für jede der folgenden vier Sprachen, dass sie unentscheidbar ist:
	\begin{enumerate}
		\item $\textsc{Halt} := \{\langle M,w\rangle \;|\; M \text{ hält auf Eingabe } w\}$
		\item $\textsc{Diverge} := \{\langle M,w\rangle \;|\; M \text{ hält nicht auf Eingabe } w\}$
		\item $\textsc{NeverHalt} := \{\langle M \rangle \;|\; \textsc{Halt}(M) = \varnothing\}$
		\item $\textsc{AlwaysHalt} := \{\langle M \rangle \;|\; \textsc{Halt}(M) = \Sigma^{\ast}\}$
	\end{enumerate}
\end{exercise}

\section*{Donnerstag}

\begin{exercise}[Entscheidbarkeit II][\href{https://moodle.studiumdigitale.uni-frankfurt.de/moodle/mod/assign/view.php?id=245311}{Moodle}\athome\medium]\
	Wir beschränken uns im Folgenden auf deterministische Turingmaschinen.
	% ALGO2 WiSe 20/21 - Blatt 9, Aufgabe 1b und 2
	\begin{enumerate}
		\item Zeige über eine Reduktion ausgehend von der Sprache \textsc{Accept}, dass die Sprache $$L_1 = \left\{\langle M,w\rangle \mathrel{\Bigg|} \begin{array}{l}M\text{ ist eine Turingmaschine}\\\text{und }M\text{ erreicht Zustand 1 für Eingabe }w\end{array}\right\}$$ nicht entscheidbar ist.
		
		\item Zeige unter Anwendung des Satzes von Rice, dass die folgende Sprache nicht entscheidbar ist. Begründe die Nicht-Trivialität der Menge $S$.
 $$L_2 = \left\{\langle M\rangle \mathrel{\Bigg|} \begin{array}{l}M\text{ ist eine Turingmaschine}\\\text{und } |L(M)| \leq 2\end{array}\right\}$$
			
		Hierbei ist $L(M)$ die von der deterministischen Turingmaschine $M$ akzeptierte Sprache.
	\end{enumerate}
\end{exercise}

\begin{exercise}[Entscheidbarkeit III][\atschool\medium]
	% Erickson Models of Computation - Chapter 7, Exercise 6c, i, Eigenkreation, k
	Entwerfe für jedes der folgenden Entscheidungsprobleme entweder einen Algorithmus oder beweise, dass das Problem unentscheidbar ist. Die Eingabe ist für jedes Entscheidungsproblem die Kodierung $\langle M \rangle$ einer Turingmaschine $M$.
	\begin{enumerate}
		\item Akzeptiert $M$ die Eingabe $\langle M \rangle \langle M \rangle$?
		\item Akzeptiert $M$ alle Palindrome?
		\item Akzeptiert $M$ die Sprache $\{\langle M \rangle \;|\; M \text{ hat min. 100 Zustände und hält auf Eingabe } \langle M \rangle\}$?
		\item Gibt es einen Eingabestring, der $M$ zu einer Bewegung nach links zwingt?
	\end{enumerate}
\end{exercise}

\section*{Weitere Aufgaben und Projekte \projekt}

\begin{exercise}[Entscheidbarkeit IV]\
	% Erickson Models of Computation - Chapter 7, Exercise 10a,b,g
	Argumentiere für jedes der folgenden Entscheidungsprobleme, dass es unentscheidbar ist.
	\begin{enumerate}
		\item Für ein als Eingabe gegebenes Python-Programm \texttt{fib.py} wollen wir verifizieren, ob \texttt{fib.py} ein korrektes Programm für die Fibonacci-Zahlen ist, also ob \texttt{fib.py} bei Eingabe $n$ die $n$-te Fibonacci-Zahl ausgibt.
		\item Kann ein als Eingabe gegebenes Python-Programm \texttt{p.py} in eine Endlosschleife gelangen?
		\item Berechnen zwei als Eingabe gegebene Python-Programme, \texttt{p1.py} und \texttt{p2.py}, die gleiche Funktion?
	\end{enumerate}
\end{exercise}
\begin{exercise}[Entscheidbarkeit V]
	% Erickson Models of Computation - Chapter 7, Exercise 7b,e,n
	Entwerfe für jedes der folgenden Entscheidungsprobleme entweder einen Algorithmus oder beweise, dass das Problem unentscheidbar ist. Die Eingabe ist für jedes Entscheidungsproblem die Kodierung $\langle M,w \rangle$ einer Turingmaschine $M$ und ihrem Eingabestring $w$.
	\begin{enumerate}
		\item Akzeptiert $M$ entweder $w$ oder $w^R$?
		\item Akzeptiert $M$ die Eingabe $w$ in höchstens $2^{|w|}$ Schritten?
		\item Wenn $M$ auf der Eingabe $w$ ausgeführt wird, gelangt $M$ jemals wieder in den Startzustand?
	\end{enumerate}
\end{exercise}

\end{document}
