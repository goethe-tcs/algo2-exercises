% LTeX: language=de_DE
\documentclass{uebung_cs}
\usepackage{algo223}
\uebung{4}{}{}
\blattname{Übungen zu Woche 4: Amortisierte Analyse}

%%%%%%%%%%%%%%%%%%%%%%%%%%%%%%%%%%%%%%%%%%%%%%%%%%%%%%%%%%%%%%%%%%%%%%%%%%%%
\begin{document}

\section*{Dienstag}
\begin{exercise}[Amortisierte Analyse][\href{https://moodle.studiumdigitale.uni-frankfurt.de/moodle/mod/assign/view.php?id=238780}{moodle}\athome]
    Hier ist eine Funktion \texttt{f($n$)}, die eine andere Funktion \texttt{g()} aufruft:
    \begin{lstlisting}[frame=single,frameround=tttt,xleftmargin=2em,linewidth=0.5\textwidth,numbers=none,language=Python]
def f($n$):
    for $i$ in range($n$):
        g()
\end{lstlisting}
Wir nehmen an, dass der $i$-te Aufruf von \texttt{g()} eine Laufzeit von $T(i)$ hat, wobei
\[T(i) = \begin{cases} 2i & \text{falls $i$ eine Zweierpotenz ist;} \\ 1 & \text{sonst.} \end{cases}\]
Analysiere die amortisierte Laufzeit von \texttt{g()}.
    \begin{enumerate}
        \item Benutze die Aggregationsmethode.
        \item Benutze das Buchhalter-Argument (\textit{accounting method}).
    \end{enumerate}
\end{exercise}

        
% \begin{exercise}[Amortisierte Analyse][\athome]
%     % Algorithms and Data Structures 2 - amortized.pdf
%     Wir haben eine Datenstruktur mit einer Operation \texttt{Foo()}. Die Kosten~$T(i)$ der $i$-ten Ausführung dieser Operation sind gegeben durch: 
%     \[T(i) = \begin{cases} 2i & \text{falls $i$ eine Zweierpotenz ist}, \\ 1 & \text{sonst}. \end{cases}\] 
%     Was ist die amortisierte Laufzeit der Operation \texttt{Foo()}? Benutze sowohl die Aggregationsmethode, als auch das Buchhalter-Argument (\textit{accounting method}), um die amortisierte Laufzeit zu analysieren.
% \end{exercise}

\begin{exercise}[Mengenvereinigungen]
    % Algorithms and Data Structures 2 - amortized.pdf
    Wir betrachten eine abstrakte Datenstruktur, die disjunkte Teilmengen von $\{1,\dots,n\}$ verwaltet (ganz ähnlich zur Union-Find).
    Anfangs sind die Elemente auf $n$ einzelne Mengen aufgeteilt, also $\{1\}, \{2\}, \dots,\{n\}$.
    Die Datenstruktur unterstützt die folgenden Operationen:
    \begin{itemize}
        \item Union($A,B$): Führe die beiden Mengen $A$ und $B$ zu einer neuen Menge $C = A \cup B$ zusammen und lösche die alten Mengen $A$ und $B$.
        \item SameSet($x,y$): Gebe \textit{true} zurück, wenn $x$ und $y$ in derselben Menge liegen, ansonsten gebe \textit{false} zurück.
    \end{itemize}
    Wir implementieren diese Operationen nun nicht mit \emph{Weighted Quick-Union}, sondern mithilfe einer Variante von \emph{Quick-Find}, die wir \emph{Weighted Quick-Find} nennen.
    Hierbei speichern wir wie bei \emph{Quick-Find} ein Array~$F[1\dots n]$, das jedem Element eine \enquote{Farbe} zuordnet; eine Zahl $F[i]\in\{1,\dots,n\}$ heißt hierbei, dass Element~$i$ die Farbe~$F[i]$ erhält.
    Außerdem wird ein Array $M[1\dots n]$ aufrecht erhalten, das für jede Farbe $f\in\{1,\dots,n\}$ die Einträge $M[f].\mathtt{list}$ und  $M[f].\mathtt{length}$ enthält. Hier ist $M[f].\mathtt{list}$ eine verkettete Liste aller Elemente~$i\in\{1,\dots,n\}$ mit $F[i]=f$ und $M[f].\mathtt{length}$ ist die Länge der Liste $M[f].\mathtt{list}$.
    Die Union-Operation übernimmt für alle Elemente in der kleineren Menge die Farbe der größeren Menge (bei Gleichstand wird die Farbe der Menge $A$ gewählt). Die SameSet-Operation überprüft, ob die zwei Elemente dieselbe Farbe haben.

    Wir werden nun folgendes beobachten:
    Aus Sicht der worst-case Laufzeit ist die Union-Operation von \emph{Weighted Quick-Find} viel schlechter als \emph{Weighted Quick-Union}, aber die amortisierten Kosten sind asymptotisch gleich.
    \begin{enumerate}
        \item\athome Schreibe die Datenstruktur mit ihren beiden Operationen als Code oder Pseudocode auf.
        \item\athome Analysiere die \textit{worst-case} Laufzeit der beiden Operationen.
        \item\atschool\mittel Zeige, dass die amortisierten Kosten $\Oh(\log n)$ für Union und $\O(1)$ für SameSet betragen.
        \item\atschool\note Zeige ebenfalls, dass jede Sequenz von $m$ Union-Operationen und $l$ SameSet-Operationen die worst-case Laufzeit $\Oh(m \log n + l)$ einhält.
        % \item[]
        (\emph{Tipp: \reflectbox{Wie oft kann ein Element die Farbe wechseln?}})
    \end{enumerate}
\end{exercise}

% \clearpage
\section*{Donnerstag}
\begin{exercise}[Potenzialfunktion: Verdopplung von Arrays][\href{https://moodle.studiumdigitale.uni-frankfurt.de/moodle/mod/assign/view.php?id=238781}{moodle}\athome]
    % Algorithms and Data Structures 2 - amortized.pdf
    Wir betrachten einen Stack, in dem wir nur \texttt{push} und nicht \texttt{pop} erlauben. Der Stack wird durch ein dynamisch wachsendes Array dargestellt ist, das heißt, das Array wird verdoppelt, wenn \texttt{push} aufgerufen wird und das Array voll ist.
    Sei $A_i$ das Array nach der $i$-ten Operation.
    Wir definieren die Potenzialfunktion $\Phi(A_i) = k$, wobei $k$ die aktuelle Anzahl an belegten Elementen im Array ist.
    Können wir mit dieser Potenzialfunktion zeigen, dass die amortisierten Kosten von \texttt{push} konstant sind? Wenn ja, wie? Wenn nein, warum nicht?
\end{exercise}

\begin{exercise}[Dynamische Hashtabelle]
    % Algorithms and Data Structures 2 - amortized.pdf
    Wir wollen nun eine Hashtabelle mit linearem Sondieren so implementieren, dass das verwendete Array mit der Verdopplungsmethode dynamisch wächst.
    Deine Datenstruktur soll also die folgenden Operationen unterstützen:
    \begin{itemize}
        \item \texttt{init()} initialisiert eine neue, leere Hashtabelle.
        \item \texttt{set($k,v$)} fügt das Schlüssel-Wert-Paar $(k,v)$ in die Hashtabelle ein.
        \item \texttt{get($k$)} liefert den Wert, der mit dem Schlüssel $k$ assoziiert ist, oder \texttt{null}, falls dieser nicht in der Hashtabelle enthalten ist.
    \end{itemize}
    Entwirf nun eine effiziente Datenstruktur, die nur $\Oh(n)$ Speicher verwendet, wobei $n$ die aktuelle Anzahl der Elemente in der Hashtabelle ist.
    \begin{enumerate}
        \item\athome Beschreibe deine Datenstruktur in Code oder Pseudocode.
        \item\athome Was ist die worst-case und die amortisierte Laufzeit deiner Implementierung von \texttt{set($k,v$)}?
        \item\atschool\note Manchmal ist es möglich, Datenstrukturen zu \enquote{deamortisieren}. Das heißt, man erhält dieselben \emph{worst-case} Schranken wie im amortisierten Fall, indem die teure Arbeit auf viele Operationen verteilt wird.
        Verändere nun deine Datenstruktur so, dass die \emph{worst-case} Laufzeit konstant ist, aber weiterhin nur $\Oh(n)$ Speicherplatz benutzt wird.
        (\emph{Tipp: \reflectbox{Kopiere bei jeder \texttt{set}-Operation in das neue Array}})
    \end{enumerate}
\end{exercise}%
%
\section*{Weitere Aufgaben und mögliche Projekte}

\begin{exercise}[Splay-Bäume][\bestehen]
    % Algorithms and Data Structures 2 - amortized.pdf
    Gegeben sei ein leerer Splay-Baum.
    \begin{enumerate}
        \item Füge die Schlüssel $41,38,31,12,19,8$ in der angegebenen Reihenfolge ein. Wie sieht der resultierende Splay-Baum aus?
        \item Wie sieht der Splay-Baum aus, nachdem man Schlüssel $31$ entfernt hat?
    \end{enumerate}
\end{exercise}

\begin{exercise}[Spiel-Bäume][\projekt]
    % Algorithms and Data Structures 2 - amortized.pdf
    Professor Sheldon schlägt die sogenannten Spiel-Bäume als eine einfachere Variante der Splay-Bäume vor. Hierbei verzichten wir innerhalb der \texttt{splay($x$)}-Methode auf die nervigen \emph{roller-coaster} und \emph{zig-zag} Transformationen, und führen stattdessen nur einfache Rotationen durch---so lange, bis $x$ an der Wurzel landet.
    \begin{enumerate}
        \item{}\mittel Was sind die amortisierten Kosten der Operation \texttt{splay($x$)}, wenn wir nur einfache Rotationen benutzen? Analysiere die Kosten mit derselben Potenzialfunktion, die wir für Splay-Bäume verwendet haben.
        \item{}\mittel Was sind die gesamten (tatsächlichen) Kosten, wenn man zuerst $n$ Elemente mit den Schlüsseln $1,2,3,\dots,n$ der Reihenfolge nach in einen Spiel-Baum einfügt und dann in der gleichen Reihenfolge sucht?
        \item{}\note Professor Sheldon behauptet, dass das Einfügen, Suchen und Löschen in Spiel-Bäumen amortisierte Kosten von $\Oh(\log n)$ hat. \enquote{Du musst nur eine raffiniertere Potenzialfunktion benutzen}, sagt er. Liegt er damit richtig?
    \end{enumerate}
\end{exercise}

\begin{exercise}[Queues mit zwei Stacks][\projekt]\mbox{}\\
    Wir erinnern uns, dass ein Stack \texttt{S} eine Datenstruktur ist, die die Operationen \texttt{S.PUSH(x)}, \texttt{S.POP()} und \texttt{S.ISEMPTY()} unterstützt. Wir nehmen an, dass diese Operationen in konstanter \emph{worst-case} Zeit implementiert worden sind.

    Du sollst für diese Aufgabe eine Queue \texttt{Q} mithilfe von zwei Stacks \texttt{S1}, \texttt{S2} implementieren. Dabei darfst du nur $\Oh(1)$ zusätzlichen Speicher benutzen.
    Der \textit{einzige} Zugriff auf die Stacks erfolgt als \emph{black-box} über die Standardoperationen \texttt{PUSH}, \texttt{POP} und \texttt{ISEMPTY}.
    \begin{enumerate}
        \item{}\mittel
              \textbf{Beschreibe mit Pseudocode} eine solche Queue-Implementierung, wobei die amortisierte worst-case Laufzeit für jede \texttt{Q.ENQUEUE(x)}, \texttt{Q.DEQUEUE()} und \texttt{Q.ISEMPTY()} Operation konstant sein muss.
        \item{}\note
              \textbf{Beweise mithilfe der Potenzialmethode}, dass die amortisierten Kosten deiner Implementierung tatsächlich konstant sind.
    \end{enumerate}
\end{exercise}

\begin{exercise}[Implementierung von dynamischen Tabellen][\projekt]
    % Algorithms and Data Structures 2 - amortized.pdf
    Implementiere deine eigene dynamische Tabelle für \texttt{Integer}-Werte, ohne dabei built-in Methoden zu verwenden. Deine dynamische Tabelle soll folgende Operationen beherrschen:
    \begin{itemize}
        \item Einfügen von Elementen,
        \item Löschen von Elementen,
        \item Ausgabe enthaltener Elemente,
        \item Ausgabe der Größe der Tabelle.
    \end{itemize}
\end{exercise}

\begin{exercise}[Deamortisierung für Stacks mit dynamischen Arrays][\projekt]
    % Algorithms and Data Structures 2 - amortized.pdf
    Manchmal ist es möglich, Datenstrukturen zu \enquote{deamortisieren}. Das heißt, man erhält dieselben \emph{worst-case} Schranken wie im amortisierten Fall, indem die teure Arbeit auf viele Operationen verteilt wird.

    \begin{enumerate}
        \item
        Wir betrachten einen Stack, der nur \texttt{push} unterstützen soll.
        Wie kann man \texttt{push} so implementieren, dass die \emph{worst-case} Laufzeit konstant ist, aber weiterhin nur $\Oh(n)$ Speicherplatz benutzt wird, wo $n$ die aktuelle Anzahl der gespeicherten Elemente ist?

        (\emph{Tipp: \reflectbox{Benutze die Verdopplungsmethode, aber verteile die Arbeit auf alle Einfügungen}})

        \item Jetzt mach dasselbe für Stacks, die \texttt{push} \emph{und} \texttt{pop} unterstützen.
    \end{enumerate}
\end{exercise}

\begin{exercise}[Puzzle der Woche: Prinzessinnen][\spass]
    % Algorithms and Data Structures 2 - amortized.pdf
    Stelle dir vor, du bist ein junger Prinz aus dem fernen Lande Algo und der König des benachbarten Landes Logik hat 3 Töchter. Die Älteste erzählt immer die Wahrheit, die Jüngste lügt immer und die Mittlere lügt und sagt die Wahrheit, wie es ihr gefällt.

    Du willst entweder die älteste oder die jüngste Tochter heiraten, da immer lügen genauso gut ist, wie immer die Wahrheit zu sagen. Nur die mittlere Tochter möchtest du nicht heiraten. Allerdings sehen alle Töchter gleich aus, sodass du sie nicht unterscheiden kannst.

    Der König ist ein hinterhältziger Mann und erlaubt dir, genau \textit{eine} Frage an genau \textit{eine} der drei Töchter zu stellen. Diese Frage soll mit \glqq Ja\grqq{} oder \glqq Nein\grqq{} zu beantworten sein. Danach musst du dich entscheiden, welche der drei Prinzessinnen du heiraten möchtest.

    Welche Frage solltest du stellen und dann welche der Töchter aussuchen?
\end{exercise}

\end{document}
