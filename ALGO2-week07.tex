% !TeX spellcheck = de_DE
\documentclass{uebung_cs}
\usepackage{algo221}
\uebung{7}{}{}
\blattname{Übungen zu Woche 7: Hartnäckigkeit I}

\usepackage[ruled]{algorithm2e}

%%%%%%%%%%%%%%%%%%%%%%%%%%%%%%%%%%%%%%%%%%%%%%%%%%%%%%%%%%%%%%%%%%%%%%%%%%%%
\begin{document}

Das Übungsblatt enthält alle empfohlenen Lernaktivitäten für die aktuelle Woche.

\begin{itemize}
\item \textbf{Heimarbeit bis Montag 17:00.}
    \begin{itemize}
    \item 
    Schau die Videos an und lies die Buchkapitel.
    \item Bearbeite die \emoji{seedling}-Aufgabe in \href{https://moodle.studiumdigitale.uni-frankfurt.de/moodle/course/view.php?id=2241}{Moodle}. (Feste Abgabefrist!)
    \item Lese den Aufgabentext aller Übungsaufgaben.
    \end{itemize}
\item \textbf{Heimarbeit.} Bearbeite die Übungsaufgaben soweit möglich. Probier zumindest alle mal!
\item \textbf{Dienstag/Donnerstag.}
\begin{itemize}
    \item \textbf{8:00--8:15.} Besprechung im Hörsaal.
    \item \textbf{8:15--9:15.} Bearbeite jetzt die Übungen, die du noch nicht lösen konntest. Sprich mit anderen Studis! Frag das Vorlesungsteam um Hilfe!
    \item \textbf{9:15--9:45.} Lösungsspaziergang zu den Aufgaben für heute.
\end{itemize}

\item \textbf{Heimarbeit bis Freitag, den 03.12., 17:00.} Gib deine Lösungen zu der \emoji{star}-Aufgabe von diesem Übungsblatt in \href{https://moodle.studiumdigitale.uni-frankfurt.de/moodle/course/view.php?id=2241}{Moodle} ab. (Feste Abgabefrist!)
\end{itemize}

\section*{Dienstag}

\begin{aufgabe}[NP]\
	% KT - Exercise 8.1
	Entscheide für die beiden folgenden Fragen, welche der Antworten \glqq Ja\grqq{}, \glqq Nein\grqq{} oder \glqq Unbekannt, da es $\P = \NP$ beantworten würde\grqq{} zutrifft. Erkläre kurz, warum du dich für deine Antwort entschieden hast.

	\begin{quote}
		\textsc{IntervalScheduling}: \\
		Sei eine Menge von Intervallen auf einer Zeitleiste und eine Schranke $k$ gegeben. Enthält diese Menge an Intervallen eine Teilmenge von sich nicht überschneidenden Intervallen der Größe mindestens $k$?
	\end{quote}
	
	\begin{enumerate}
		\item Gilt \textsc{IntervalScheduling} $\leq_p$ \textsc{VertexCover}?
		\item Gilt \textsc{IndependentSet} $\leq_p$ \textsc{IntervalScheduling}?
	\end{enumerate}
\end{aufgabe}


\begin{aufgabe}[Kundenanalyse]\
	% KT - Exercise 8.2
	Zur Analyse des Kundenverhaltens benutzen Geschäfte oftmals ein zweidimensionales Array $A$, wo die Zeilen die Kunden und die Spalten die verkauften Produkte enthalten. Ein Eintrag $A[i,j]$ gibt an, welche Menge von Produkt $j$ der Kunde $i$ gekauft hat, so hat Chelsea im folgenden Beispiel siebenmal Katzenstreu gekauft.
	
	\begin{center}
	\begin{tabular}{l c c c c}
	\hline 
	& Waschmittel & Bier & Windeln & Katzenstreu \\ 
	\hline 
	Raj & 0 & 6 & 0 & 3 \\ 
	Alanis & 2 & 3 & 0 & 0 \\ 
	Chelsea & 0 & 0 & 0 & 7 \\ 
	\hline 
	\end{tabular} 
	\end{center}
	
	Ein Geschäft möchte nun eine \textit{diverse} Teilmenge der Kunden finden, weil das nützlich sein kann, um Marktforschung zu betreiben. Eine Teilmenge $S$ der Kunden heißt \textit{divers}, wenn keine zwei Kunden aus $S$ dieselben Produkte gekauft haben. Mit anderen Worten gibt es für jedes Produkt nur genau einen Kunden aus $S$, der es gekauft hat. Das lässt sich als das Problem \textsc{DiverseTeilmenge} formulieren.
	\begin{quote}
		\textsc{DiverseTeilmenge}:\\
		Seien ein $n \times m$ Array $A$, wie oben definiert, und eine Zahl $k$ mit $k \leq n$ gegeben. Gibt es eine \textit{diverse} Teilmenge $S$ der Kunden mit $|S| \geq k$?
	\end{quote}
	Zeige, dass \textsc{DiverseTeilmenge} \NP-vollständig ist.
\end{aufgabe}    

\begin{aufgabe}[Feriencamp]
	% KT - Exercise 8.3
	Bei der Organisation eines Feriencamps wirst du mit dem Problem \textsc{EffizientesRekrutieren} konfrontiert.
	\begin{quote}
		\textsc{EffizientesRekrutieren}:\\
		Für jede der $n$ im Feriencamp angebotenen Sportarten muss es einen für diesen Sport geschulten Betreuer geben. Es haben sich $m$ potenzielle Betreuer beworben. Für jede der $n$ Sportarten gibt es eine Teilmenge der $m$ Bewerber, die für diese Sportart qualifiziert sind. Ist es möglich, maximal $k < m$ Betreuer einzustellen, sodass mindestens ein Betreuer für jede Sportart geschult ist?
	\end{quote}	 
	
	Zeige, dass \textsc{EffizientesRekrutieren} \NP-vollständig ist.
\end{aufgabe}

\section*{Donnerstag}

\begin{aufgabe}[Ressourcenmanagement]
	% KT - Exercise 8.4
	Ein Unternehmen verwaltet High-Performance Echtzeitsysteme mit asynchronen Prozessen, die auf gemeinsame Ressourcen zugreifen müssen. Das System hat eine Menge von $n$ Prozessen und eine Menge von $m$ Ressourcen. 
	Zu jedem Zeitpunkt spezifiziert ein Prozess eine Menge an Ressourcen, die er benutzen möchte. Jede Ressource kann gleichzeitig von mehreren Prozessoren angefragt werden, aber es kann zu jedem Zeitpunkt nur \emph{genau} ein Prozess auf die Ressource zugreifen. 
	
	Deine Aufgabe ist es, die Ressourcen an die Prozesse zu verteilen. Wenn ein Prozess alle benötigten Ressourcen zugeteilt bekommt, ist er \textit{aktiv}, ansonsten ist er \textit{blockiert}. Eine effiziente Zuweisung der Ressourcen sorgt dafür, dass möglichst viele Prozesse \textit{aktiv} sind, was sich als das Problem \textsc{RessourcenReservieren} ausdrücken lässt.
	 \begin{quote}
	 \textsc{RessourcenReservieren}:\\
	 Es sei eine Menge von $n$ Prozessen, eine Menge von $m$ Ressourcen und eine Zahl $k$ gegeben. Ist es möglich, die Ressourcen den Prozessen so zuzuweisen, dass mindestens $k$ Prozesse \textit{aktiv} sind?
	 \end{quote}
	 
	Gib für jedes Problem auf der folgenden Liste entweder einen Polynomialzeit-Algorithmus an oder zeige, dass das Problem \NP-vollständig ist.
	 \begin{enumerate}
	 	\item \textsc{RessourcenReservieren}.
	 	\item \textsc{RessourcenReservieren} mit dem Spezialfall $k=2$.
	 	\item \textsc{RessourcenReservieren} mit zwei Typen von Ressourcen. Hierbei sind die Ressourcen: Mensch und Maschine. Jeder Prozess benötigt höchstens eine Ressource von jedem Typ. (Anders ausgedrückt braucht jeder Prozess einen bestimmten Menschen und eine bestimmte Maschine.)
	 	\item \textsc{RessourcenReservieren} mit dem Spezialfall, dass jede Ressource von maximal zwei Prozessen angefragt wird.	 	
	 \end{enumerate}
\end{aufgabe}

\begin{aufgabe}[Clique]
	% Algorithms and Data Structures 2 - NP.pdf
	Für einen ungerichteten Graphen $G = (V,E)$  ist eine \textit{Clique} eine Teilmenge $V' \subseteq V$ der Knotenmenge, sodass alle Knoten in $V'$ benachbart sind, also wenn zwischen jedem Knotenpaar $v,w \in V'$ mit $v \neq w$ eine Kante $(v,w) \in E$ existiert. Ein Graph $G$ hat eine $k$-\textit{Clique}, wenn $|V'| = k$ gilt. Das Entscheidungsproblem \textsc{Clique} ist wie folgt definiert:
	\begin{quote}
		\textsc{Clique}:\\
		\textit{Input:} Ein ungerichteter Graph $G = (V,E)$ und ein $k \in \mathbb{N}_0$.\\
		\textit{Output:} \glqq JA\grqq{}, wenn $G$ eine $k$-\textit{Clique} besitzt, sonst \glqq NEIN\grqq{}.
	\end{quote}
	\begin{enumerate}
		\item Zeige, dass \textsc{Clique} $\in \NP$.
		\item Zeige, dass \textsc{Clique} \NP-vollständig ist, indem du \textsc{IndependentSet} auf \textsc{Clique} reduzierst.
	\end{enumerate}
\end{aufgabe}

\section*{Sternaufgabe}

\begin{aufgabe}[\emoji{star}:]
	
\end{aufgabe}

\end{document}
