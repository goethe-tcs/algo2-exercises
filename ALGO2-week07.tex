% !TeX spellcheck = de_DE
\documentclass{uebung_cs}
\usepackage{algo221}
\uebung{7}{}{}
\blattname{Übungen zu Woche 7: Hartnäckigkeit I}

\usepackage[ruled]{algorithm2e}
%\usepackage{circuitikz}

%%%%%%%%%%%%%%%%%%%%%%%%%%%%%%%%%%%%%%%%%%%%%%%%%%%%%%%%%%%%%%%%%%%%%%%%%%%%
\begin{document}

Das Übungsblatt enthält alle empfohlenen Lernaktivitäten für die aktuelle Woche.

\begin{itemize}
\item \textbf{Heimarbeit bis Montag 17:00.}
    \begin{itemize}
    \item 
    Schau die Videos an und lies die Buchkapitel.
    \item Bearbeite die \emoji{seedling}-Aufgabe in \href{https://moodle.studiumdigitale.uni-frankfurt.de/moodle/course/view.php?id=2241}{Moodle}. (Feste Abgabefrist!)
    \item Lese den Aufgabentext aller Übungsaufgaben.
    \end{itemize}
\item \textbf{Heimarbeit.} Bearbeite die Übungsaufgaben soweit möglich. Probier zumindest alle mal!
\item \textbf{Dienstag/Donnerstag.}
\begin{itemize}
    \item \textbf{8:00--8:15.} Besprechung im Hörsaal.
    \item \textbf{8:15--9:15.} Bearbeite jetzt die Übungen, die du noch nicht lösen konntest. Sprich mit anderen Studis! Frag das Vorlesungsteam um Hilfe!
    \item \textbf{9:15--9:45.} Lösungsspaziergang zu den Aufgaben für heute.
\end{itemize}

\item \textbf{Heimarbeit bis Freitag, den 03.12., 17:00.} Gib deine Lösungen zu der \emoji{star}-Aufgabe von diesem Übungsblatt in \href{https://moodle.studiumdigitale.uni-frankfurt.de/moodle/course/view.php?id=2241}{Moodle} ab. (Feste Abgabefrist!)
\end{itemize}

\section*{Dienstag}

\begin{aufgabe}[Schaltkreise]
	% Eigenkreation
	In der Vorlesung haben wir eine Reduktion von \textsc{CircuitSat} auf \textsc{3SAT} gesehen. Gegeben sei folgender Schaltkreis:
	
	%\begin{center}
	%\begin{circuitikz} 
	%\ctikzset{logic ports=ieee}
%		\draw
%		(0,2)  node[and port] (and_1) {}
%		(0,0)  node[or port]  (or_1)  {}
%		(0,-2) node[not port] (not_1) {}
%		(3,1)  node[not port] (not_2) {}
%		(3,-1) node[and port] (and_2) {}
%		(6,0)  node[or port]  (or_2)  {}		
%		(0,2) node[and port] (myand1) {}
%		(0,0) node[and port] (myand2) {}
%		(2,1) node[or port] (myxnor) {}
%		(and_1.out) -| (not_2.in)
%		(or_1.out)  -| (and_2.in 1)
%		(not_1.out) -| (and_2.in 2)
%		(and_2.out) -| (or_2.in 2)
%		(not_2.out) -| (or_2.in 1)
%		(or_2.out) -- ++(1,0) node[near end,above]{Out}
		
%		(and_1.in 1) -- ++(-1.5,0)node[left](In1){x}
%		(and_1.in 2) -- ++(-1.5,0)node[left](In2){y}
%		(or_1.in 1) -- (In2)
%		(not_1.in) -- ++(-1.5,0)node[left](In3){z}
%		(or_1.in 2) -- (In3);
%		(myand1.out) -- (myxnor.in 1)
%		(myand2.out) -- (myxnor.in 2);
%	\end{circuitikz}
%	\end{center}
	
	Welche \textsc{3CNF} Formel gibt die Reduktion aus, wenn die Eingabe aus dem dargestellten Schaltkreis besteht?\\
\end{aufgabe}

\begin{aufgabe}[DNF Erfüllbarkeit]
	% Erickson, Chapter 12, Exercise 3
	Eine aussagenlogische Formel ist in \textit{disjunktiver Normalform} (DNF), wenn sie eine Disjunktionen (\textsc{Or}) von Konjunktionstermen (\textsc{And}) ist. Ein Beispiel für eine DNF ist:
	$$(\overline{x} \wedge y \wedge \overline{z}) \vee (y \wedge z) \vee (x \wedge \overline{y} \wedge \overline{z}).$$
	Gegeben ist eine aussagenlogische Formel in disjunktiver Normalform. DNF-\textsc{Sat} entscheidet, ob diese Funktion erfüllbar ist.
	\begin{enumerate}
		\item Beschreibe einen Algorithmus, der DNF-\textsc{Sat} in Polynomialzeit  löst.\\
		\item Was ist der Fehler im folgenden Argument, in dem $\P = \NP$ gezeigt wird?
		\begin{quote}
			Angenommen, wir haben eine aussagenlogische Formel in konjunktiver Normalform mit höchstens 3 Literalen pro Klausel. Wir wollen herausfinden, ob diese Funktion erfüllbar ist. Wir können das Distributivgesetz für Boolesche Operationen verwenden, um eine äquivalente Formel in disjunktiver Normalform zu konstruieren. Zum Beispiel:
			$$(x \vee y \vee \overline{z}) \wedge (\overline{x} \vee \overline{y}) \Leftrightarrow (x \wedge \overline{y}) \vee (y \wedge \overline{x}) \vee (\overline{z} \wedge \overline{x}) \vee (\overline{z} \wedge \overline{y}).$$
			Nun können wir den Algorithmus aus Aufgabenteil a) verwenden, um in Polynomialzeit herauszufinden, ob die entstandene DNF erfüllbar ist. Wir haben also \textsc{3Sat} in Polynomialzeit gelöst! Da \textsc{3Sat} $\NP$-hart ist, gilt $\P = \NP$!	
		\end{quote}
	\end{enumerate}	 
\end{aufgabe}

\begin{aufgabe}[Perfektes Matching]\
	% https://courses.engr.illinois.edu/cs374/fa2021/A/labs/lab12.pdf - Exercise 1
	Sei $G = (V,E)$ ein Graph. Eine Teilmenge $M \subseteq E$ ist ein \textit{Matching}, wenn keine zwei Kanten aus $M$ mit dem selben Knoten verbunden sind. Ein Matching ist \textit{perfekt}, wenn jeder Knoten $v \in V$ zu einer Kante $m \in M$ inzident ist. Eine andere mögliche Definition betrachtet die Größe des Matchings: $M$ ist perfekt, wenn $|M| = |V|/2$ gilt. Wir definieren das \textsc{PerfektesMatching}-Problem:
	\begin{quote}
		\textsc{PerfektesMatching}:\\
		Gegeben sei ein Graph $G = (V,E)$. Existiert in $G$ ein perfektes Matching?
	\end{quote}
	Dieses Problem kann in Polynomialzeit gelöst werden. Das ist ein fundamentales Ergebnis der kombinatorischen Optimisierung mit vielen Anwendungen in Theorie und Praxis. Es stellt sich heraus, dass das \textsc{PerfektesMatching}-Problem auf bipartiten Graphen einfacher zu lösen ist. Ein Graph $G = (V,E)$ ist \textit{bipartit}, wenn die Menge der Knoten $V$ auf zwei Teilmengen $L$ und $R$ verteilt werden können, sodass alle Kanten zwischen $L$ und $R$ liegen. In anderen Worten: $L$ und $R$ sind unabhängige Knotenmengen (Independent Set). Im Folgenden ist ein Algorithmus vorgeschlagen, der allgemeine Graphen auf bipartite Graphen reduziert:
	\begin{quote}
		\textsc{ReduceGraph}:\\
		Gegeben sei ein Graph $G = (V,E)$. Erstelle daraus einen bipartiten Graphen $H = (V \times \{1,2\},E_H)$ wie folgt:\\
		Jeder Knoten $u \in V$ wird zu zwei Kopien $(u,1)$ und $(u,2)$. Dabei ist $V_1 = \{(u,1)|u\in V\}$ eine Seite und $V_2 = \{(u,2)|u\in V\}$ die andere Seite des bipartiten Graphen. Sei $E_H = \{((u,1),(u,2))|(u,v) \in E\}$. In anderen Worten: Wenn $(u,v) \in E$, fügen wir eine Kante zwischen $(u,1)$ und $(v,2)$ ein. Beachte, dass es keine Eigenschleifen in $G$ gibt und es dadurch in $H$ für alle $u \in V$ keine Kante  $((u,1),(u,2))$ geben kann.
	\end{quote}
	Ist die vorgeschlagene Reduktion korrekt? Um das herauszufinden müssen wir überprüfen, ob $H$ nur genau dann ein perfektes Matching hat, wenn $G$ eins hat.
	\begin{enumerate}
		\item Zeige: Wenn $G$ ein perfektes Matching hat, dann hat auch $H$ eins.
		\item Sei $G$ der vollständige Graph mit $3$ Knoten, also ein Dreieck. Zeige: $G$ hat kein perfektes Matching, aber $H$ hat eins.
		\item Erweitere das Beispiel aus der vorherigen Teilaufgabe: Konstruiere einen Graphen $G$ mit einer gerade Anzahl an Knoten, sodass $G$ kein perfektes Matching hat, $H$ aber schon.
	\end{enumerate}
	Damit ist die Reduktion inkorrekt, obwohl eine Richtung wahr ist.
\end{aufgabe}

\begin{aufgabe}[Independent Set]\
	% https://courses.engr.illinois.edu/cs374/fa2021/A/labs/lab12.pdf - Exercise 2
	Ein Independent Set in einem Graphen $G = (V,E)$ ist eine Teilmenge $S \subseteq V$, sodass keine zwei Knoten $u,v \in S$ durch eine Kante $(u,v) \in E$ verbunden sind. Du hast eine magische Blackbox, die auf wundersame Art und Weise das folgende Entscheidungsproblem in Polynomialzeit beantwortet:
	\begin{itemize}[topsep=0.21cm, leftmargin=1.2cm]
		\item \textsc{Input}: Ein ungerichteter Graph $G$ und eine Zahl $k \in \N$.
		\item \textsc{Output}: \textsc{True}, wenn $G$ ein Independent Set der Größe $k$ besitzt, sonst \textsc{False}.
	\end{itemize}
	\begin{enumerate}
		\item Benutze die Blackbox, um einen Algorithmus zu beschreiben, der folgendes Optimisierungsproblem in Polynomialzeit löst:
		
		\begin{itemize}[topsep=0.21cm, leftmargin=1.2cm]
			\item \textsc{Input}: Ein ungerichteter Graph $G$.
			\item \textsc{Output}: Die Größe des größten Independent Set in $G$.
		\end{itemize}
		\item Benutze die Blackbox, um einen Algorithmus zu beschreiben, der folgendes Suchproblem in Polynomialzeit löst:
		
		\begin{itemize}[topsep=0.21cm, leftmargin=1.2cm]
			\item \textsc{Input}: Ein ungerichteter Graph $G$.
			\item \textsc{Output}: Ein Independent Set in $G$ mit maximaler Größe.
		\end{itemize}
	\end{enumerate}
\end{aufgabe}

\section*{Donnerstag}

\begin{aufgabe}[NP]\
	% KT - Exercise 8.1
	Entscheide für die beiden folgenden Fragen, welche der Antworten \glqq Ja\grqq{}, \glqq Nein\grqq{} oder \glqq Unbekannt, da es $\P = \NP$ beantworten würde\grqq{} zutrifft. Erkläre kurz, warum du dich für deine Antwort entschieden hast.

	\begin{quote}
		\textsc{IntervalScheduling}: \\
		Sei eine Menge von Intervallen auf einer Zeitleiste und eine Schranke $k$ gegeben. Enthält diese Menge an Intervallen eine Teilmenge von sich nicht überschneidenden Intervallen der Größe mindestens $k$?
	\end{quote}
	
	\begin{enumerate}
		\item Gilt \textsc{IntervalScheduling} $\leq_p$ \textsc{VertexCover}?
		\item Gilt \textsc{IndependentSet} $\leq_p$ \textsc{IntervalScheduling}?
	\end{enumerate}
\end{aufgabe}


\begin{aufgabe}[Kundenanalyse]\
	% KT - Exercise 8.2
	Zur Analyse des Kundenverhaltens benutzen Geschäfte oftmals ein zweidimensionales Array $A$, wo die Zeilen die Kunden und die Spalten die verkauften Produkte enthalten. Ein Eintrag $A[i,j]$ gibt an, welche Menge von Produkt $j$ der Kunde $i$ gekauft hat, so hat Chelsea im folgenden Beispiel siebenmal Katzenstreu gekauft.
	
	\begin{center}
	\begin{tabular}{l c c c c}
	\hline 
	& Waschmittel & Bier & Windeln & Katzenstreu \\ 
	\hline 
	Raj & 0 & 6 & 0 & 3 \\ 
	Alanis & 2 & 3 & 0 & 0 \\ 
	Chelsea & 0 & 0 & 0 & 7 \\ 
	\hline 
	\end{tabular} 
	\end{center}
	
	Ein Geschäft möchte nun eine \textit{diverse} Teilmenge der Kunden finden, weil das nützlich sein kann, um Marktforschung zu betreiben. Eine Teilmenge $S$ der Kunden heißt \textit{divers}, wenn keine zwei Kunden aus $S$ dieselben Produkte gekauft haben. Mit anderen Worten gibt es für jedes Produkt nur genau einen Kunden aus $S$, der es gekauft hat. Das lässt sich als das Problem \textsc{DiverseTeilmenge} formulieren.
	\begin{quote}
		\textsc{DiverseTeilmenge}:\\
		Seien ein $n \times m$ Array $A$, wie oben definiert, und eine Zahl $k$ mit $k \leq n$ gegeben. Gibt es eine \textit{diverse} Teilmenge $S$ der Kunden mit $|S| \geq k$?
	\end{quote}
	Zeige, dass \textsc{DiverseTeilmenge} \NP-vollständig ist.
\end{aufgabe}    

\begin{aufgabe}[Feriencamp]
	% KT - Exercise 8.3
	Bei der Organisation eines Feriencamps wirst du mit dem Problem \textsc{EffizientesRekrutieren} konfrontiert.
	\begin{quote}
		\textsc{EffizientesRekrutieren}:\\
		Für jede der $n$ im Feriencamp angebotenen Sportarten muss es einen für diesen Sport geschulten Betreuer geben. Es haben sich $m$ potenzielle Betreuer beworben. Für jede der $n$ Sportarten gibt es eine Teilmenge der $m$ Bewerber, die für diese Sportart qualifiziert sind. Ist es möglich, maximal $k < m$ Betreuer einzustellen, sodass mindestens ein Betreuer für jede Sportart geschult ist?
	\end{quote}	 
	
	Zeige, dass \textsc{EffizientesRekrutieren} \NP-vollständig ist.
\end{aufgabe}


\section*{Sternaufgabe}

\begin{aufgabe}[\emoji{star}: Search-to-Decision]\
	% Ericksen, Chapter 12, Exercise 5b,c,d
	\begin{enumerate}
		\item Du besitzt eine magische Blackbox, die in Polynomialzeit herausfindet, wie viele Knoten ein größter vollständiger Teilgraph eines beliebigen Graphen $G$ besitzt. Beschreibe und analysiere einen Algorithmus, der in Polynomialzeit für einen beliebigen Graphen $G$ einen vollständigen Teilgraphen maximaler Größe berechnet. Benutze dafür die Blackbox.
		
		\item Du besitzt eine magische Blackbox, die in Polynomialzeit herausfindet, ob ein beliebiger Graph $G$ \textit{$3$-färbbar} ist. Beschreibe und analysiere einen Algorithmus, der in Polynomialzeit für einen beliebigen Graphen $G$ eine richtige $3$-Färbung ausgibt oder richtigerweise ausgibt, dass keine solche Färbung existiert. Benutze dafür die Blackbox.\\
		\textit{Tipp: Die Eingabe für die Blackbox ist ein Graph und nichts Anderes.}
		
		\item Du besitzt eine magische Blackbox, die in Polynomialzeit herausfindet, ob eine beliebige aussagenlogische Formel $\Phi$ erfüllbar ist. Beschreibe und analysiere einen Algorithmus, der in Polynomialzeit eine erfüllende Belegung der Variablen von $\Phi$ ausgibt, oder richtigerweise ausgibt, dass so eine Belegung nicht existiert. Benutze dafür die Blackbox.
	\end{enumerate}
\end{aufgabe}

\end{document}
