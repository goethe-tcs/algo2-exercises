% !TeX spellcheck = de_DE
\documentclass{uebung_cs}
\usepackage{algo221}
\uebung{13}{}{}
\blattname{Übungen zu Woche 13: Lineare Programmierung II}

\usepackage[ruled]{algorithm2e}
\usepackage{cancel}
\usepackage[smaller]{acronym}
\newacro{LP}{Linear Program}
\newacro{ILP}{Integer Linear Program}

%%%%%%%%%%%%%%%%%%%%%%%%%%%%%%%%%%%%%%%%%%%%%%%%%%%%%%%%%%%%%%%%%%%%%%%%%%%%
\begin{document}

Das Übungsblatt enthält alle empfohlenen Lernaktivitäten für die aktuelle Woche.

\begin{itemize}
\item \textbf{Heimarbeit bis Montag 17:00.}
    \begin{itemize}
    \item 
    Schau die Videos an und lies die Buchkapitel.
    \item Bearbeite die \emoji{seedling}-Aufgabe in \href{https://moodle.studiumdigitale.uni-frankfurt.de/moodle/course/view.php?id=2241}{Moodle}. (Feste Abgabefrist!)
    \item Lese den Aufgabentext aller Übungsaufgaben.
    \end{itemize}
\item \textbf{Heimarbeit.} Bearbeite die Übungsaufgaben soweit möglich. Probier zumindest alle mal!
\item \textbf{Dienstag/Donnerstag.}
\begin{itemize}
    \item \textbf{8:00--8:15.} Besprechung im Hörsaal.
    \item \textbf{8:15--9:15.} Bearbeite jetzt die Übungen, die du noch nicht lösen konntest. Sprich mit anderen Studis! Frag das Vorlesungsteam um Hilfe!
    \item \textbf{9:15--9:45.} Lösungsspaziergang zu den Aufgaben für heute.
\end{itemize}
\end{itemize}

\section*{Dienstag}

\begin{aufgabe}[Primal vs Dual \acs{LP}]
	%Write down the corresponding dual problem to the following \acs{LP}.
	Stelle für das folgende \acs{LP} das korrespondierende duale \acs{LP} auf.
	\[
		\begin{array}{rrrlllllllll}
		\text{max}    &     &   2 x_1       &   +     &   x_2   &       &         &   -   & 3x_4  &         &   \\
		\text{s.t.}  &     &   2x_1      &   +     &   4x_2  &   +   &  3 x_3   &   -   & 2x_4   &  \geq   & 1 \\
							&     &   5x_1      &   -     &   2x_2   &   +   &   x_3  &   -   & 4x_4  &  \leq   & 2 \\
							&     &   -3x_1     & +     &   x_2  &   -   &   6x_3  &   -   & 7 x_4   &   =     & 0 \\
							&     &           &       &         &       &       &         &x_1,x_3   & \leq    &   0      \\
							&     &          &       &         &       &       &         &  x_4       & \geq    &   0    
		\end{array}
	\]
	
\end{aufgabe}

\begin{aufgabe}[Bipartites Maximum Matching \acs{LP}]\
	% Erickson Extended Dance Remix - Chapter H, Exercise 2a,b
%	The \emph{Bipartite Maximum Matching Problem} can be formulated as a \acs{LP}. The input is a bipartite graph $G=(U\cup V;E)$ with $E \subseteq U \times V$ and the output is the largest matching in $G$.
%	\begin{enumerate}
%		\item Provide a linear program that has one variable for each edge.
%		\item Derive the dual of the \acs{LP} of \emph{part a)}. What do the dual variables represent? What does the objective function represent? What problem is this!?
%	\end{enumerate}
	Das \emph{Bipartite Maximum Matching Problem} kann als \acs{LP} formuliert werden. Die Eingabe ist ein bipartiter Graph $G = (U \cup V, E)$, wobei $E \subseteq U \times V$ und die Ausgabe ist das größte Matching in $G$.
	\begin{enumerate}
		\item Stelle ein lineares Programm auf, das für jede Kante eine Variable besitzt.
		\item Stelle das duale \acs{LP} aus \emph{Teil a)} auf. Was repräsentieren die dualen Variablen und die Zielfunktion? Welches Problem ist das?
	\end{enumerate}
\end{aufgabe}

\begin{aufgabe}[Totale Unimodularität]
	% Erickson Extended Dance Remix - Chapter I, Exercise 9a,b
%	A minor of a matrix $A$ is the submatrix defined by any subset of the rows and any subset of the columns. A matrix $A$ is \emph{totally unimodular} if, for every square minor $M$, the determinant of $M$ is $−1, 0$, or $1$.
%	\begin{enumerate}
%		\item Let A be an arbitrary totally unimodular matrix.
%		\begin{itemize}
%			\item[i.] Prove that the transposed matrix $A^T$ is also totally unimodular.
%			\item[ii.] Prove that negating any row or column of A leaves the matrix totally unimod- ular.
%			\item[iii.] Prove that the block matrix $[A | I ]$ is totally unimodular.
%		\end{itemize}
%		\item Prove that for any totally unimodular matrix $A$ and any integer vector $b$, the canonical linear program $\max\{c^T x  \; | \; Ax \leq b, x \geq 0\}$ has an integer optimal solution. \\
%		\emph{Hint: Cramer’s rule.}
%	\end{enumerate}
	Der Minor (auch Unterdeterminante genannt) einer Matrix $A$ ist die Determinante einer quadratischen Untermatrix, die aus jeder Teilmenge der Zeilen und jeder Teilmenge der Spalten von $A$ bestehen kann. Eine Matrix A ist \emph{total unimodular}, wenn jeder Minor $M$ von $A$ den Wert $-1,0$ oder $1$ hat.
	\begin{enumerate}
		\item Sei $A$ eine beliebige total unimodulare Matrix.
		\begin{enumerate}
			\item[i.] Beweise, dass die transponierte Matrix $A^T$ ebenfalls total unimodular ist.
			\item[ii.] Beweise, dass die Matrix $A$ durch das Negieren beliebiger Zeilen oder Spalten total unimodular bleibt.
			\item[iii.] Beweise, dass die Blockmatrix $[\;A\;|\;I\;]$ total unimodular ist.
		\end{enumerate}
		\item Beweise, dass das kanonische lineare Programm $\max\{c^T x  \; | \; Ax \leq b, x \geq 0\}$ für eine beliebige total unimodulare Matrix $A$ und einen beliebigen ganzzahligen Vektor $b$ eine ganzzahlige optimale Lösung hat.

		\emph{Hinweis: Cramersche Regel.}
	\end{enumerate}
	
\end{aufgabe}

\section*{Donnerstag}


\begin{aufgabe}[Schulzuweisungs \acs{ILP}]
%	Consider a school district with $I$ schools, $G$ grades at each school and $J$ neighborhoods. Each school $i$ has a capacity of $C_{ig}$ for grade $g$. In each neighborhood $j$, the student population of grade $g$ is $S_{jg}$. Finally, the distance of school $i$ from neighborhood $j$ is $d_{ij}$. Formulate an integer linear programming problem whose objective is to assign all students to schools, while minimizing the total distance traveled by all students.\\
%	\emph{Hint: introduce variables $x_{jig}$, which represent the number of students from neighborhood $j \in J$, who visit school $i \in I$ in grade $g \in G$}
	Betrachte einen Schulbezirk mit $I$ Schulen, $G$ Jahrgangsstufen an jeder Schule und $J$ Nachbarschaften. Jede Schule $i$ hat eine Kapazität von $C_{ig}$ für den Jahrgang $g$. Die Anzahl der Schüler der Jahrgangsstufe $g$ aus jeder Nachbarschaft $j$ beträgt $S_{jg}$. Die Distanz einer Schule $i$ zu Nachbarschaft $j$ beträgt $d_{ij}$.
	
	Formuliere ein ganzzahliges lineares Programm, dessen Zielfunktion es ist, alle Schüler einer Schule zuzuweisen, wobei die von allen Schülern zurückgelegte Distanz zwischen ihrer Nachbarschaft und ihrer Schule minimiert werden soll.

	\emph{Hinweis: Führe Variablen $x_{jig}$ ein, die die Anzahl der Schüler aus Nachbarschaft $j \in J$, die die Schule $i \in I$ in der Jahrgangsstufe $g \in G$ besuchen, darstellen.}
\end{aufgabe}


\begin{aufgabe}[Schief-symmetrische Matrizen]\
	% Erickson Extended Dance Remix - Chapter H, Exercise 1a,b
%	A matrix $A=(a_{ij})$ is \emph{skew-symmetric} if and only if $a_{ji} = −a_{ij}$ for all indices $i \neq j$; in particular, every skew-symmetric matrix is square. A canonical linear program $\max\{c^T x  \; | \; Ax \leq b, x \geq 0\}$ is self-dual if the matrix $A$ is skew-symmetric and the objective vector $c$ is equal to the constraint vector $b$.
%	\begin{enumerate}
%		\item Prove that any self-dual linear program (P) is syntactically equivalent to its dual linear program (D).
%		\item Show that any linear program (P) with $n$ variables and $m$ constraints can be transformed into a self-dual linear program with $m + n$ variables and $m + n$ constraints. The optimal solution to the self-dual program should include both the optimal solution for (P) (in $n$ of the variables) and the optimal solution for the dual program (D) (in the other $m$ variables).
%	\end{enumerate}
	Eine Matrix $A = (a_{ij})$ wird genau dann \emph{schief-symmetrisch} genannt, wenn $a_{ji} = -a_{ij}$ für alle Indizes $i, j$ gilt; insbesondere ist jede schief-symmetrische Matrix quadratisch. Ein kanonisches lineares Programm ${\max\{c^T x  \; | \; Ax \leq b, x \geq 0\}}$ ist selbst-dual, wenn die Matrix $A$ schief-symmetrisch ist und für den Zielvektor $c$ und den Vektor der Nebenbedingungen $b$ folgender Zusammenhang besteht: $c = -b$.
	
	% \begin{enumerate}
	Beweise, dass ein beliebiges selbst-duales lineares Programm (P) syntaktisch äquivalent zu seinem dualen linearen Programm (D) ist.
	% 	\item Zeige, dass jedes beliebige lineare Programm (P) mit $n$ Variablen und $m$ Nebenbedingungen in ein selbst-duales lineares Programm mit $m + n$ Variablen und $m + n$ Nebenbedingungen transformiert werden kann. Die optimale Lösung zum selbst-dualen Programm sollte sowohl die optimale Lösung für (P) (in $n$ der Variablen), als auch die optimale Lösung für das duale Programm (D) (in den anderen $m$ Variablen) beinhalten.
	% \end{enumerate}
\end{aufgabe}

\begin{aufgabe}[\acp{ILP} sind NP-hart]
%	Let $A \in \Z^{m\times n}, b \in \Z^m$, $P := \{x \in \R^n : Ax \leq b \}$.
%	The Integer Feasibility problem is to decide if $P \cap \Z^n \neq \emptyset$.
%	Show that it is NP-hard using a reduction from 3-SAT. \\
%	\emph{Hint: You want a transformation of a 3-CNF formula with $m$ clauses over $n$ variables to $P$.}
	Sei $A \in \Z^{m \times n}, b \in \Z^m$ und $P := \{x \in \R^n : Ax \leq b\}$. Das \emph{Integer Feasability Problem} ist, zu entscheiden, ob $\P \cap \Z^n \neq \emptyset$ gilt. 
	Zeige, dass das \emph{Integer Feasability Problem} \NP-hart ist, indem du eine Reduktion ausgehend von \textsc{3-Sat} durchführst.

	\emph{Hinweis: Es soll eine \textsc{3CNF} Formel mit $m$ Klauseln und $n$ Variablen zu $P$ transformiert werden.}
\end{aufgabe}


\end{document}
