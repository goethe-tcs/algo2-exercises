% !TeX spellcheck = de_DE
\documentclass{uebung_cs}
\usepackage{algo221}
\uebung{5}{}{}
\blattname{Übungen zu Woche 5: Randomisierte Algorithmen I}

%%%%%%%%%%%%%%%%%%%%%%%%%%%%%%%%%%%%%%%%%%%%%%%%%%%%%%%%%%%%%%%%%%%%%%%%%%%%
\begin{document}

Das Übungsblatt enthält alle empfohlenen Lernaktivitäten für die aktuelle Woche.

\begin{itemize}
\item \textbf{Heimarbeit bis Montag 17:00.}
    \begin{itemize}
    \item 
    Schau die Videos an und lies die Buchkapitel.
    \item Bearbeite die \emoji{seedling}-Aufgabe in \href{https://moodle.studiumdigitale.uni-frankfurt.de/moodle/course/view.php?id=2241}{Moodle}. (Feste Abgabefrist!)
    \item Lese den Aufgabentext aller Übungsaufgaben.
    \end{itemize}
\item \textbf{Heimarbeit.} Bearbeite die Übungsaufgaben soweit möglich. Probier zumindest alle mal!
\item \textbf{Dienstag/Donnerstag.}
\begin{itemize}
    \item \textbf{8:00--8:15.} Besprechung im Hörsaal.
    \item \textbf{8:15--9:15.} Bearbeite jetzt die Übungen, die du noch nicht lösen konntest. Sprich mit anderen Studis! Frag das Vorlesungsteam um Hilfe!
    \item \textbf{9:15--9:45.} Lösungsspaziergang zu den Aufgaben für heute.
\end{itemize}

\item \textbf{Heimarbeit bis Freitag, den 19.11., 17:00.} Gib deine Lösungen zu der \emoji{star}-Aufgabe von diesem Übungsblatt in \href{https://moodle.studiumdigitale.uni-frankfurt.de/moodle/course/view.php?id=2241}{Moodle} ab. (Feste Abgabefrist!)
\end{itemize}

\section*{Dienstag}

\begin{aufgabe}[Wahrscheinlichkeitsrechnung]
	% Algorithms and Data Structures 2 - randomizedI.pdf
	Löse die folgenden Aufgaben.
	\begin{enumerate}
		\item Seien $E$ und $F$ zwei Ereignisse, so dass $\Pr(E | F) = \Pr(E)$ und $\Pr(F | E) = P(F)$. Zeige, dass $\Pr(E \cap F) = \Pr(E) \cdot \Pr(F)$. \\
		\item Betrachte die \textit{contention resolution analysis of failure} für mindestens einen Prozess. Wenn wir die \textit{union bound} anwenden, sind die Ereignisse dann disjunkt oder überschneiden sie sich?
	\end{enumerate}		
\end{aufgabe}

\begin{aufgabe}[Contention Resolution]
	% Algorithms and Data Structures 2 - randomizedI.pdf
	Führe das \textit{contention resolution} Protokoll mit 4 Prozessen aus. Benutze zwei Münzen oder eine zufällige Nummer, um die zufällige Auswahl zu simulieren. Wie viele Runden braucht man, bis alle Prozesse erfolgreich auf die Datenbank zugegriffen haben?
\end{aufgabe}    

\begin{aufgabe}[Minimum cut]
	% Algorithms and Data Structures 2 - randomizedI.pdf
	Betrachte das \glqq kleine Graph\grqq{} Beispiel zum \textit{contraction algorithm}, das wir in der Vorlesung behandelt haben. Löse die folgenden Aufgaben.
	\begin{enumerate}
		\item Zeige, dass es eine Auswahl von Kontraktionen gibt, die zu einem nicht minimalen Schnitt führen.\\
		\item Berechne die Wahrscheinlichkeit, dass der Kontraktionsalgorithmus den minimalen Schnitt ausgibt.
	\end{enumerate}
\end{aufgabe}

\section*{Donnerstag}

\begin{aufgabe}[Schneller Kontraktionsalgorithmus]
	% Algorithms and Data Structures 2 - randomizedI.pdf
	Zeige, wie man den Kontraktionsalgorithmus für minimale Schnitte effizient implementiert.
\end{aufgabe}

\begin{aufgabe}[Kontraktionsalgorithmus analysieren]
	% Algorithms and Data Structures 2 - randomizedI.pdf
	Betrachte die Wahrscheinlichkeitsanalyse des Erfolgs für den Kontraktionsalgorithmus. Die folgende Herleitung führt zum gewünschten Ergebnis.
	\begin{align}
	\Pr(E_{n-2} \cap \cdots \cap E_1) &= \Pr(E_{n-3} \cap \cdots \cap E_1) \cdot \Pr(E_{n-2}|E_{n-3} \cap \cdots \cap E_1) \\
	&= \Pr(E_1) \cdot \Pr(E_2|E_1) \cdot \Pr(E_{j+1} | E_j \cap \cdots \cap E_1) \cdots \Pr(E_{n-2} | E_{n-3} \cap \cdots \cap E_1) \\
	&= \left(1 - \frac{2}{n} \right) \cdot \left(1 - \frac{2}{n-1} \right) \cdot \left(1 - \frac{2}{n-2} \right) \cdots \left(1 - \frac{2}{3} \right) \\
	&= \left(\frac{n-2}{n}\right) \cdot \left(\frac{n-3}{n-1}\right) \cdot \left(\frac{n-4}{n-2}\right) \cdots \left(\frac{2}{4}\right) \cdot \left(\frac{1}{3}\right) \\
	&= \left(\frac{2}{n(n-1)}\right) \geq \frac{2}{n^2}
	\end{align}
	Löse die folgenden Aufgaben.
	\begin{enumerate}
		\item Zeige (1) auf zwei Wege: (I) benutze die Formel für die bedingte Wahrscheinlichkeit und (II) interpretiere, was die linke und rechte Seite repräsentieren. \\
		\item Zeige (2). \textit{Hinweis:} Benutze das Ergebnis aus Teilaufgabe a). \\
		\item Füge die Grenzen für jeden Faktor ein, um (3) zu zeigen. \\
		\item Zeige (4). \\
		\item Zeige (5).
	\end{enumerate}
\end{aufgabe}

\begin{aufgabe}[Mehrheit]
	% Algorithms and Data Structures 2 - randomizedI.pdf
	Gegeben ist eine Sequenz $x_1,x_2,\dots,x_n$ von $n$ Integern. Die Sequenz hat ein \textit{Mehrheitselement} $t$, wenn eine Nummer $t$ öfters als $\frac{n}{2}$-mal in der Sequenz vorkommt. Zum Beispiel hat die Sequenz $1,2,3,1,2,2,2$ das Mehrheitselement $2$, während die Sequenz $2,2,1,2,3,3$ kein Mehrheitselement hat.
	
	Hier ist ein randomisierter Algorithmus für das Problem: Wähle zufällig eine Zahl $x_i$ aus der Sequenz. Prüfe dann, ob dieses Element öfters als $\frac{n}{2}$-mal in der Sequenz vorkommt. Wenn ja, ist es das Mehrheitselement. Wenn das nicht der Fall ist, gibt der Algorithmus aus, dass es kein Mehrheitselement gibt. Löse die folgenden Aufgaben.
	\begin{enumerate}
		\item Was ist die Laufzeit des Algorithmus? \\
		\item Kann der Algorithmus ein Mehrheitselement zurückgeben, wenn die Sequenz keins hat? Begründe deine Antwort. \\
		\item Kann der Algorithmus ausgeben, dass kein Mehrheitselement existiert, obwohl die Sequenz ein solches besitzt? Begründe deine Antwort. \\
		\item Bestimme die Wahrscheinlichkeit für falsche Antworten.
	\end{enumerate}
\end{aufgabe}

\section*{Sternaufgabe}

\end{document}
