% LTeX: language=de_DE
\documentclass{uebung_cs}
\usepackage{algo223}
\uebung{12}{}{}
\blattname{Übungen zu Woche 12: Lineare Programmierung I}

\usepackage[ruled]{algorithm2e}
\usepackage{cancel}
\usepackage[smaller]{acronym}
\newacro{LP}{Linear Program}

%%%%%%%%%%%%%%%%%%%%%%%%%%%%%%%%%%%%%%%%%%%%%%%%%%%%%%%%%%%%%%%%%%%%%%%%%%%%
\begin{document}

\section*{Dienstag}

\begin{exercise}[Lösungsmenge eines \acsp{LP}][\href{https://moodle.studiumdigitale.uni-frankfurt.de/moodle/mod/assign/view.php?id=245947}{Moodle}\athome]\
%	Consider the following \acs{LP}.
%	\[
%		\begin{array}{rrllll}
%			\text{max} &   2 x_1 +5 x_2 \\	
%			\text{s.t.} 
%			&  x_1 + 2x_2 &\leq 8\\
%			&  x_1  &\leq 4\\
%			& 2x_2 &\leq 7\\
%			& x_1,x_2&\geq  0\\
%		\end{array}
%	\] 
%
%	\begin{enumerate}
%	\item Draw the corresponding polyhedron $P$.
%	\item Find an optimal fractional and integer solution graphically.
%	\item If we add the first constraint multiplied by $p_1=1$, to the second constraint multiplied by $p_2=1$, and to the third constraint multiplied by $p_3=1.5$, we obtain 
%	$$2x_1+5x_2\leq 1\cdot 8+ 1\cdot 4+1.5\cdot 7= 22.5.$$ 
%	Find non-negative values for $p_1,p_2,p_3$ such that proceeding as above we obtain $2x_1+5x_2\leq opt$ for every $x \in P$,  where $opt$ is one of the optimal values determined in \textit{part b)}.
%	\end{enumerate}
	Betrachte das folgende \acs{LP}, welches in kanonischer Form gegeben ist.
	\[
		\begin{array}{rrllll}
			\text{max} &   2 x_1 +5 x_2 \\	
			\text{s.t.} 
			&  x_1 + 2x_2 &\leq 8\\
			&  x_1  &\leq 4\\
			& 2x_2 &\leq 7\\
			& x_1,x_2&\geq  0\\
		\end{array}
	\] 
	
	\begin{enumerate}
		\item\easy Zeichne das korrespondierende Polytop $P$ in der Ebene.
		\item\easy Finde grafisch eine optimale fraktionale und eine optimale ganzzahlige Lösung.
		\item\medium Addiert man die erste Nebenbedingung, multipliziert mit $p_1 = 1$, zu der zweiten Nebenbedingung, multipliziert mit $p_2 = 1$, und zu der dritten Nebenbedingung, multipliziert mit $p_3 = 1.5$, dann erhält man
		$$2x_1+5x_2\leq 1\cdot 8+ 1\cdot 4+1.5\cdot 7= 22.5.$$
		Finde nicht-negative Werte für $p_1, p_2, p_3$, sodass man, wenn man wie oben vorgeht, $2x_1 + 5x_2 \leq \mathrm{opt}$ für jedes $x \in P$ erhält, wobei $\mathrm{opt}$ einer der in \text{Aufgabenteil b)} bestimmten optimalen Werte ist.
	\end{enumerate}
\end{exercise}

\begin{exercise}[Standardform][\href{https://moodle.studiumdigitale.uni-frankfurt.de/moodle/mod/assign/view.php?id=245947}{Moodle}\athome]\
%	Optimization Lecture
%	Consider the following \acs{LP}.
%	\[
%		\begin{array}{rrrllll}
%			\text{max}    & 3x_1 		    & + & 2 x_2	& +  & 4x_3  &\\
%			\text{s.t.}	&  x_1		    & 	&		& +	 & 2x_3  & \le 50 \\
%								& 3x_1 		    & + & 10x_2	& +	 & 6x_3	 & \le 300\\
%								& 18 x_1 	    &   &       & +  & x_3 	 & \le 40\\ 
%								& x_1,      	&	& x_2,	&	 & x_3	 &\geq 0 
%		\end{array}
%	\]
%	\begin{enumerate}
%		\item Convert the \acs{LP} into standard form by introducing slack variables $s_1,s_2, s_3$, one for each equation.
%		\item List all the feasible bases for this \acs{LP}.   
%	\end{enumerate}
	Betrachte das folgende \acs{LP}, welches in kanonischer Form gegeben ist.
	\[
		\begin{array}{rrrllll}
			\text{max}    & 3x_1 		    & + & 2 x_2	& +  & 4x_3  &\\
			\text{s.t.}	&  x_1		    & 	&		& +	 & 2x_3  & \le 50 \\
								& 3x_1 		    & + & 10x_2	& +	 & 6x_3	 & \le 300\\
								& 18 x_1 	    &   &       & +  & x_3 	 & \le 40\\ 
								& x_1,      	&	& x_2,	&	 & x_3	 &\geq 0 
		\end{array}
	\]
	\begin{enumerate}
		\item\easy Überführe das \acs{LP} in seine Standardform, indem du für jede der Gleichungen Schlupfvariablen (\textit{slack variables}) $s_1,s_2,s_3$ einführst.
		\item\medium Zähle alle zulässigen (\emph{feasible}) Basen für das \acs{LP} auf.
	\end{enumerate}
\end{exercise}

\newpage
\begin{exercise}[Single-Source Shortest Path \acs{LP}][\atschool]
	% Erickson Extended Dance Remix - Chapter H, Exercise 7a - c
%	The \emph{single-source shortest path problem} can be formulated as a \acs{LP}, with one variable $d_v$ for each vertex $v \neq s$ in the input graph:
%
%	\[
%		\begin{array}{rrl}
%			\text{max}   &  \sum_{v} d_v & 	   \\
%			\text{s.t.}  &		   d_v & \leq l_{s \rightarrow v} \\
%						 &	 d_v - d_u & \leq l_{u \rightarrow v} \\
%						 &	       d_v & \geq 0
%		\end{array}
%	\]
%
%	Describe the behavior of the simplex algorithm on this linear program in terms of distances. Assume that the edge weights $l_{u \rightarrow v}$ are all non-negative and that there is a unique shortest path between any two vertices in the graph.
%	\begin{enumerate}
%		\item What is a basis for this linear program? What is a feasible basis? What is a locally optimal basis?
%		\item Show that in the optimal basis, every variable $d_v$ is equal to the shortest-path distance from $s$ to $v$.
%		\item Describe the primal simplex algorithm for the shortest-path linear program directly in terms of vertex distances. In particular, what does it mean to pivot from a feasible basis to a neighboring feasible basis, and how can we execute such a pivot quickly?
%	\end{enumerate}
	Das \emph{Single-Source Shortest Path Problem} kann als \acs{LP} formuliert werden, welches für jeden Knoten $v\in V(G)$ im Eingabegraphen~$G$ mit Quelle~$s\in V(G)$ eine Variable $d_v$ hat:
	\[
		\begin{array}{rrl}
			\text{max}   &  \sum_{v} d_v & 	   \\
			\text{s.t.}  &		   d_s & \leq 0\\
						 &	 d_v - d_u & \leq l_{u \rightarrow v} \quad\text{für alle $uv\in E(G)$}\\
						 &	       d_v & \geq 0\quad\text{für alle $v\in V(G)$}
		\end{array}
	\]
	
	Beschreibe das Verhalten des Simplex-Algorithmus auf diesem linearen Programm in Bezug auf die Distanzen. Nimm an, dass die Kantengewichte $l_{u \rightarrow v}$ nicht-negativ sind und dass zwischen allen Paaren von Knoten jeweils ein eindeutiger kürzester Weg existiert.
	\begin{enumerate}
		\item\medium Was ist eine Basis für dieses lineare Programm? Was ist eine zulässige Basis? Was ist eine lokal optimale Basis?
		\item\hard Zeige, dass in einer optimalen Basis jede Variable $d_v$ den gleichen Wert hat, wie die kürzeste-Pfad-Distanz von $s$ zu $v$.
		\item\medium Beschreibe den Simplex-Algorithmus für das Kürzeste-Wege lineare Programm direkt in Bezug auf die Knotendistanzen. Was bedeutet es insbesondere, wenn man eine Pivot-Operation von einer zulässigen Basis zu einer anderen zulässigen Basis ausführt?
	\end{enumerate}
	
\end{exercise}

\section*{Donnerstag}

\vspace{-0.5cm}
\begin{exercise}[Initiale Basis für Simplex][\href{https://moodle.studiumdigitale.uni-frankfurt.de/moodle/mod/assign/view.php?id=245948}{Moodle}\athome]
	% Erickson Extended Dance Remix - Chapter I, Exercise 4a,b
%	In this exercise, we develop another standard method for computing an initial feasible basis for the primal simplex algorithm. Suppose we are given a linear program (P) with $d$ variables and $n + d$ constraints as input:
%
%	\[
%		\begin{array}{rrl}
%			\text{max}   &  cx &  	    \\
%			\text{s.t.}  &	Ax & \leq b \\
%						 &	x  & \geq 0
%		\end{array}
%	\]
%
%	To compute an initial feasible basis for (P), we solve a modified linear program (P') defined by introducing a new variable $\lambda$ and two new constraints $0 \geq \lambda \geq 1$, and modifying the objective function:
%
%	\[
%		\begin{array}{rrl}
%			\text{max}   &        \lambda & 	   \\
%			\text{s.t.}  &	Ax - b\lambda & \leq 0 \\
%						 &		  \lambda & \leq 1 \\
%						 &	 x, \lambda   & \geq 0
%		\end{array}
%	\]
%
%	\begin{enumerate}
%		\item Prove that $x_1 = x_2 = \dots = x_d = \lambda = 0$ is a feasible basis for (P).
%		\item Prove that (P) is feasible, if and only if the optimal value for (P') is 1.
%	\end{enumerate}
	In dieser Aufgabe entwickeln wir einen Algorithmus, um eine initiale zulässige (\textit{feasible}) Basis zu berechnen, mit der der Simplex-Algorithmus anfangen kann. Nimm an, wir erhalten ein lineares Programm $(P)$ mit $m$ Variablen und $n + m$ Nebenbedingungen als Eingabe:
	
	\[
		\begin{array}{rrl}
			\text{max}   &  c^T x &  	    \\
			\text{s.t.}  &	Ax & \leq b \\
						 &	x  & \geq 0
		\end{array}
	\]
	
	Um eine initiale zulässige Basis für $(P)$ zu berechnen, lösen wir ein modifiziertes lineares Programm $(P')$, für das eine initiale zulässige Basis immer leicht zu finden ist. In $(P')$ gibt es eine neue Variable $\lambda$ und zwei neue Nebenbedingungen $0 \leq \lambda \leq 1$, und eine neue Zielfunktion:
	\[
		\begin{array}{rrl}
			\text{max}   &        \lambda & 	   \\
			\text{s.t.}  &	Ax - b\lambda & \leq 0 \\
						 &		  \lambda & \leq 1 \\
						 &	 x, \lambda   & \geq 0
		\end{array}
	\]
	
	\begin{enumerate}
		\item\easy Beweise, dass $x_1 = x_2 = \dots = x_d = \lambda = 0$ immer eine zulässige Basis für $(P')$ ist.
		\item\medium Beweise, dass $(P)$ genau dann eine Lösung hat, wenn für $(P')$ der optimale Wert $1$ ist.
		\item\hard Beschreibe in 1-3 Sätzen, wie wir eine initiale zulässige Basis für $(P)$ berechnen können.
	\end{enumerate}
\end{exercise}

\begin{exercise}[Polyeder][\atschool]
	% Erickson Extended Dance Remix - Chapter I, Exercise 2a, Optimization Lecture
%	Let $P=\{ x\in \mathbb{R}^n : Ax\leq b \}$ be a polyhedron, where $A\in \mathbb{R}^{m\times n}$ and $b\in \mathbb{R}^m$.
%	\begin{enumerate}
%		\item Give an example of a non-empty $P$ that is unbounded for every objective vector $c$.
%		\item Prove that $P$ is convex.
%	\end{enumerate}
	Sei $P=\{ x\in \mathbb{R}^n : Ax\leq b \}$ ein Polyhedron, wobei $A\in \mathbb{R}^{m\times n}$ und $b\in \mathbb{R}^m$.
	\begin{enumerate}
		\item\easy Gib ein Beispiel für ein nicht-leeres $P$, das für jeden Zielvektor $c$ unbeschränkt ist.
		\item\hard Beweise, dass $P$ konvex ist.
	\end{enumerate}
\end{exercise}

\begin{exercise}[Lokal optimale und zulässige Basis][\atschool]\
	% Erickson Extended Dance Remix - Chapter I, Exercise 1a,b
%	Two bases are \emph{neighbors} if they have $d − 1$ constraints in common. Equivalently, in geometric terms, two vertices are neighbors if they lie on a line determined by some $d − 1$ constraint hyperplanes.
%
%	A basis is \emph{locally optimal} if its location $x$ is the optimal solution to the linear program with the same objective function and only the constraints in the basis. Geometrically, a basis is locally optimal if its location $x$ is the lowest point in the intersection of those $d$ halfspaces.
%
%	Fix a non-degenerate linear program in canonical form with d variables and n + d constraints.
%	\begin{enumerate}
%		\item Prove that every feasible basis has exactly d feasible neighbors.
%		\item Prove that every locally optimal basis has exactly n locally optimal neighbors.
%	\end{enumerate}
	% Zwei Basen sind \emph{Nachbarn}, wenn sie $d-1$ Nebenbedingungen gemeinsam haben. Äquivalent dazu sind zwei Knoten geometrisch gesehen Nachbarn, wenn sie auf einer Linie liegen, die durch $d-1$ Hyperebenen aus Nebenbedingungen bestimmt wird.
	%
	% Eine Basis ist \emph{lokal optimal}, wenn ihre Lage $x$ die optimale Lösung zu einem linearen Programm ist, das dieselbe Zielfunktion und nur die Nebenbedingungen in der Basis hat. Geometrisch gesehen ist eine Basis lokal optimal, wenn ihre Lage $x$ der niedrigste Punkt im Schnittpunkt der $d$ Halbräume ist.
	Gegeben sei ein nicht-degenerierendes lineares Programm in kanonischer Form mit $n$ Variablen und $m + n$ Nebenbedingungen auf.
	\begin{enumerate}
		\item\hard Argumentiere, dass jede zulässige Basis genau $n$ zulässige (\textit{feasible}) Nachbarn hat.
		\item\hard Argumentiere, dass jede lokal optimale Basis genau $m$ lokal optimale Nachbarn hat.
		
		\emph{Hinweis: Wandle das LP in eine äquivalente Form um.}
	\end{enumerate}
\end{exercise}

%\newpage

\section*{Weitere Aufgaben und Projekte}

\begin{exercise}[Carath{\'e}odory's Theorem][\hard]
%	Let $A_1,\dots , A_n$ be a collection of $n$ vectors in $\R^m$.
%	\begin{enumerate}
%		\item Show that any element of
%			\[
%				C=\left\{\sum_{i=1}^n \lambda_i A_i : \lambda_1, \dots , \lambda_n\geq 0\right\}
%			\]
%			can be expressed as $\sum_{i=1}^n \lambda_i A_i$, with $\lambda_i\geq 0$ and at most $m$ of the coefficients $\lambda_i$ being non-zero. 
%
%			\textit{Hint: Consider the polyhedron
%			\[
%				\Lambda=\left\{(\lambda_1, \dots , \lambda_n)\in \R^n : \sum_{i=1}^n\lambda_i A_i = y,\ \lambda_1, \dots , \lambda_n\geq 0\right\}.
%			\]}
%		\item Let $P$ be the convex hull of the vectors $A_i$, given by
%			\[
%				P=\left\{\sum_{i=1}^n \lambda_i A_i : \sum_{i=1}^n\lambda_i=1,\ \lambda_1, \dots , \lambda_n\geq 0\right\}.
%			\]
%			Show that any element of $P$ can be expressed as $\sum_{i=1}^n \lambda_i A_i$, where $\sum_{i=1}^n \lambda_i=1$ and $\lambda_i\geq 0$ for all $i$, with at most $m+1$ of the coefficients $\lambda_i$ being nonzero.
%	\end{enumerate}
	Seien $A_1,\dots , A_n$ Vektoren in $\R^m$.
	\begin{enumerate}
		\item Zeige, dass jedes Element~$y$ von 
			\[
				C=\left\{\sum_{i=1}^n \lambda_i A_i \;:\; \lambda_1, \dots , \lambda_n\geq 0\right\}
			\]
			als eine Summe $y=\sum_{i=1}^n \lambda_i A_i$ geschrieben werden kann, in der $\lambda_i\geq 0$ für alle Koeffizienten gilt und höchstens $m$ der Koeffizienten $\lambda_i$ von Null verschieden sind. 

			\textit{Hinweis: Betrachte das Polyhedron
			\[
				\Lambda=\left\{(\lambda_1, \dots , \lambda_n)\in \R^n \;:\; \sum_{i=1}^n\lambda_i A_i = y,\ \lambda_1, \dots , \lambda_n\geq 0\right\}.
			\]}
		\item Sei $P$ die konvexe Hülle der Vektoren $A_1,\dots,A_n$, gegeben durch
			\[
				P=\left\{\sum_{i=1}^n \lambda_i A_i \;:\; \sum_{i=1}^n\lambda_i=1,\ \lambda_1, \dots , \lambda_n\geq 0\right\}.
			\]
			Zeige, dass jedes Element $y\in P$ als eine Summe $y=\sum_{i=1}^n \lambda_i A_i$ geschrieben werden kann, in der $\lambda_i\geq 0$ für alle Koeffizienten gilt, $\sum_{i=1}^n\lambda_i=1$ gilt, und höchstens $m+1$ der Koeffizienten $\lambda_i$ von Null verschieden sind. 
	\end{enumerate}
\end{exercise}

\end{document}
